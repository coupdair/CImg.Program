index: \begin{itemize}
\item \hyperlink{index_sectionCommandLine}{command line options} \item \hyperlink{index_sectionDisplayDocumentation}{documentation outline} \item \hyperlink{index_sectionDoxygenSyntax}{make documentation using Doxygen syntax}\end{itemize}
\hypertarget{index_sectionCommandLine}{}\section{command line options}\label{index_sectionCommandLine}


\begin{VerbInclude}\begin{verbatim}
 CImg.display : display image as 2D map or/and 1D graph(s) or 3D surface. (Dec  2 2010, 23:35:25)

    -i               image.png                Input image
    --blur           -1                       Variance of gaussian pre-blurring (e.g. --blur 2.0)
    --norm           false                    normalize image between 0 and 255 for 8-bit display (e.g. --norm true)
    --1D             false                    display also profile (i.e. 1D graph with --1D true)
    --3D             false                    display as 3D surface
    -o               0                        Output 3D object
    -z               0.25                     Aspect ratio along z-axis
    -di              10                       Step for isophote skipping
    -pi              0                        Input 3D pose  (e.g. pose.cimg)
    -po              pose.cimg                Output 3D pose (e.g. pose.cimg)
    --render-type    2                        type of rendering (E[0-6], see GUI: render type)
    --color-type     true                     type of surface color: true image color, false constant color
    -h               true                     display this help.
\end{verbatim}
\end{VerbInclude}
\hypertarget{index_sectionDisplayDocumentation}{}\section{documentation outline}\label{index_sectionDisplayDocumentation}
This is the reference documentation of \href{http://www.meol.cnrs.fr/}{\tt CImg.display}, from the \href{http://www.univ-lille1.fr/lml/}{\tt LML}.\par
\par
 CImg.display is a image displaying software. The program begins in the main function in the \href{CImg_8display_8cpp.html}{\tt CImg.display.cpp} source file.\par
\par
 This documentation has been automatically generated from the CImg.display sources, using the tool \href{http://www.doxygen.org}{\tt doxygen}. It should be readed as HTML, LaTex and man page.\par
 It contains both \begin{itemize}
\item a detailed description of all classes and functions \item a \href{user.html}{\tt user guide}\end{itemize}
that as been documented in the sources.

\begin{Desc}
\item[Additional needed libraries:]\end{Desc}
\begin{itemize}
\item \href{http://cimg.sourceforge.net}{\tt the CImg Library} (v1.3.9) using \href{http://www.imagemagick.org/}{\tt ImageMagick} for a few image format\end{itemize}
\begin{Desc}
\item[Optional libraries:]\end{Desc}
\begin{itemize}
\item i/o \href{http://www.unidata.ucar.edu/software/netcdf/}{\tt NetCDF} (network Common Data Form) \item added to CImg raw, \href{http://www.rd-vision.com/}{\tt Hiris}, \href{http://www.pco.de/}{\tt PCO} and \href{http://www.lavision.de}{\tt LaVision} images support \item i/o \href{http://www.libpng.org/}{\tt libPNG} (Portable Network Graphics) using \href{http://www.zlib.net/}{\tt zLib} (non destructive compression) \item i/o \href{http://www.libtiff.org/}{\tt libTIFF} (Tag Image File Format) - need libJPEG for static linking \item v vector image \href{http://libboard.sourceforge.net/}{\tt Board} (A vector graphics C++ library: Postscript, SVG and Fig files)\end{itemize}
\hypertarget{index_sectionDoxygenSyntax}{}\section{make documentation using Doxygen syntax}\label{index_sectionDoxygenSyntax}
Each function in the source code should be commented using {\bf doxygen} {\bf syntax} in the same file. The documentation need to be written before the function. The basic syntax is presented in this part. 

\begin{VerbInclude}\begin{verbatim}//!This comment will be both in declaration and explanation of the function
/**
 * This comment will be in explanation part only
**/
\end{verbatim}
\end{VerbInclude}


Two kind of comments are needed for {\bf declaration} and {\bf explanation} {\bf parts} of the function: Standart documentation should the following ({\bf sample} of code documentation): 

\begin{VerbInclude}\begin{verbatim}//!BothInDeclarationNExplanation
/** 
 * InExplanationPartOnly
 * \param arg1 = ExplanationOnThisFunctionArgument
 * \param arg2 = ExplanationOnThisFunctionArgument
 *
 * \code
 *   AnExampleOfUse
 * \endcode
 *
 * \see AnOtherFunctionOrAvariable
**/
\end{verbatim}
\end{VerbInclude}


In both declaration and explanation part, {\bf writting} and {\bf highlithing} syntax can be the following:\par
\par
 \begin{itemize}
\item {\tt $\backslash$code} to get\par
\end{itemize}
\begin{itemize}
\item {\tt $\backslash$n} a new line \item {\tt $\backslash$li} a list (dot list)\end{itemize}
\begin{itemize}
\item {\tt $\backslash$b} bold style \item {\tt $\backslash$c} code style \item {\tt $\backslash$e} enhanced style (italic)\end{itemize}
For making {\bf shortcut} please use:\par
 \begin{itemize}
\item {\tt $\backslash$see} to make a shortcut to a related function or variable \item {\tt $\backslash$link} to make a shortcut to a file or a function \begin{Desc}
\item[Note:]this keyword needs to be closed using {\tt $\backslash$end$\ast$} \end{Desc}
\item {\tt $\backslash$todo} to add a thing to do in the list of \href{todo.html}{\tt ToDo} for the whole program\end{itemize}
In explanation part, {\bf paragraph} style can be the following:\par
 \begin{itemize}
\item {\tt $\backslash$code} for an example of the function use \item {\tt $\backslash$note} to add a few notes \item {\tt $\backslash$attention} for SOMETHING NOT FULLY DEFINED YET \item {\tt $\backslash$warning} to give a few warning on the function \begin{Desc}
\item[Note:]these keywords need to be closed using {\tt $\backslash$end$\ast$} \end{Desc}


\begin{VerbInclude}\begin{verbatim}\code
  WriteYourExample
\endcode
\end{verbatim}
\end{VerbInclude}
\end{itemize}
Many other keywords are defined, so please read the documentation of \href{http://www.doxygen.org/commands.html}{\tt doxygen}. 