index\-: \begin{DoxyItemize}
\item \hyperlink{index_sectionCommandLine}{command line options} \item \hyperlink{index_sectionDisplayDocumentation}{documentation outline} \item \hyperlink{index_sectionDoxygenSyntax}{make documentation using Doxygen syntax}\end{DoxyItemize}
\hypertarget{index_sectionCommandLine}{}\section{command line options}\label{index_sectionCommandLine}

\begin{DoxyVerbInclude}

 0;31;59mCImg.display: display image as 2D map or/and 1D graph(s) or 3D surface. (Feb  8 2018, 16:54:58)

    -i               image.png                Input image
    --2x2D           false                    display 2 images
    -i1              image1.png               Input image 1 (e.g. image1.png)
    -i2              image2.png               Input image 2 (e.g. image2.png)
    --blur           -1                       Variance of gaussian pre-blurring (e.g. --blur 2.0)
    --norm           false                    normalize image between 0 and 255 for 8-bit display (e.g. --norm true)
    --1D             false                    display also profile (i.e. 1D graph with --1D true)
    --3D             false                    display as 3D surface
    -o               0                        Output 3D object
    -z               0.25                     Aspect ratio along z-axis
    -di              10                       Step for isophote skipping
    -pi              0                        Input 3D pose  (e.g. pose.cimg)
    -po              pose.cimg                Output 3D pose (e.g. pose.cimg)
    --render-type    2                        type of rendering (E[0-6], see GUI: render type)
    --color-type     true                     type of surface color: true image color, false constant color
    -h               true                     display this help.
\end{DoxyVerbInclude}
\hypertarget{index_sectionDisplayDocumentation}{}\section{documentation outline}\label{index_sectionDisplayDocumentation}
This is the reference documentation of \href{http://www.meol.cnrs.fr/}{\tt C\-Img.\-display}, from the \href{http://www.univ-lille1.fr/lml/}{\tt L\-M\-L}.\par
\par
 C\-Img.\-display is a image displaying software. The program begins in the main function in the \href{CImg_8display_8cpp.html}{\tt C\-Img.\-display.\-cpp} source file.\par
\par
 This documentation has been automatically generated from the C\-Img.\-display sources, using the tool \href{http://www.doxygen.org}{\tt doxygen}. It should be readed as H\-T\-M\-L, La\-Tex and man page.\par
 It contains both \begin{DoxyItemize}
\item a detailed description of all classes and functions \item a \href{user.html}{\tt user guide}\end{DoxyItemize}
that as been documented in the sources.

\begin{DoxyParagraph}{Additional needed libraries\-:}

\end{DoxyParagraph}
\begin{DoxyItemize}
\item \href{http://cimg.sourceforge.net}{\tt the C\-Img Library} (v1.\-3.\-9) using \href{http://www.imagemagick.org/}{\tt Image\-Magick} for a few image format\end{DoxyItemize}
\begin{DoxyParagraph}{Optional libraries\-:}

\end{DoxyParagraph}
\begin{DoxyItemize}
\item i/o \href{http://www.unidata.ucar.edu/software/netcdf/}{\tt Net\-C\-D\-F} (network Common Data Form) \item added to C\-Img raw, \href{http://www.rd-vision.com/}{\tt Hiris}, \href{http://www.pco.de/}{\tt P\-C\-O} and \href{http://www.lavision.de}{\tt La\-Vision} images support \item i/o \href{http://www.libpng.org/}{\tt lib\-P\-N\-G} (Portable Network Graphics) using \href{http://www.zlib.net/}{\tt z\-Lib} (non destructive compression) \item i/o \href{http://www.libtiff.org/}{\tt lib\-T\-I\-F\-F} (Tag Image File Format) -\/ need lib\-J\-P\-E\-G for static linking \item v vector image \href{http://libboard.sourceforge.net/}{\tt Board} (A vector graphics C++ library\-: Postscript, S\-V\-G and Fig files)\end{DoxyItemize}
\hypertarget{index_sectionDoxygenSyntax}{}\section{make documentation using Doxygen syntax}\label{index_sectionDoxygenSyntax}
Each function in the source code should be commented using {\bfseries doxygen} {\bfseries syntax} in the same file. The documentation need to be written before the function. The basic syntax is presented in this part. 
\begin{DoxyVerbInclude}
//!This comment will be both in declaration and explanation of the function
/**
 * This comment will be in explanation part only
**/
\end{DoxyVerbInclude}


Two kind of comments are needed for {\bfseries declaration} and {\bfseries explanation} {\bfseries parts} of the function\-: Standart documentation should the following ({\bfseries sample} of code documentation)\-: 
\begin{DoxyVerbInclude}
//!BothInDeclarationNExplanation
/** 
 * InExplanationPartOnly
 * \param arg1 = ExplanationOnThisFunctionArgument
 * \param arg2 = ExplanationOnThisFunctionArgument
 *
 * \code
 *   AnExampleOfUse
 * \endcode
 *
 * \see AnOtherFunctionOrAvariable
**/
\end{DoxyVerbInclude}


In both declaration and explanation part, {\bfseries writting} and {\bfseries highlithing} syntax can be the following\-:\par
\par
 \begin{DoxyItemize}
\item {\ttfamily \textbackslash{}code} to get\par
\end{DoxyItemize}
\begin{DoxyItemize}
\item {\ttfamily \textbackslash{}n} a new line \item {\ttfamily \textbackslash{}li} a list (dot list)\end{DoxyItemize}
\begin{DoxyItemize}
\item {\ttfamily \textbackslash{}b} bold style \item {\ttfamily \textbackslash{}c} code style \item {\ttfamily \textbackslash{}e} enhanced style (italic)\end{DoxyItemize}
For making {\bfseries shortcut} please use\-:\par
 \begin{DoxyItemize}
\item {\ttfamily \textbackslash{}see} to make a shortcut to a related function or variable \item {\ttfamily \textbackslash{}link} to make a shortcut to a file or a function \begin{DoxyNote}{Note}
this keyword needs to be closed using {\ttfamily \textbackslash{}end$\ast$} 
\end{DoxyNote}
\item {\ttfamily \textbackslash{}todo} to add a thing to do in the list of \href{todo.html}{\tt To\-Do} for the whole program\end{DoxyItemize}
In explanation part, {\bfseries paragraph} style can be the following\-:\par
 \begin{DoxyItemize}
\item {\ttfamily \textbackslash{}code} for an example of the function use \item {\ttfamily \textbackslash{}note} to add a few notes \item {\ttfamily \textbackslash{}attention} for S\-O\-M\-E\-T\-H\-I\-N\-G N\-O\-T F\-U\-L\-L\-Y D\-E\-F\-I\-N\-E\-D Y\-E\-T \item {\ttfamily \textbackslash{}warning} to give a few warning on the function \begin{DoxyNote}{Note}
these keywords need to be closed using {\ttfamily \textbackslash{}end$\ast$} 
\end{DoxyNote}

\begin{DoxyVerbInclude}
\code
  WriteYourExample
\endcode
\end{DoxyVerbInclude}
\end{DoxyItemize}
Many other keywords are defined, so please read the documentation of \href{http://www.doxygen.org/commands.html}{\tt doxygen}. 